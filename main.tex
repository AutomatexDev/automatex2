% !TEX root = main.tex
% !TEX program = pdflatex

\documentclass[portugues]{automatex}

\usepackage{lipsum}

\title{Título do trabalho de conclusão de curso}
\shorttitle{Título do trabalho de conclusão de curso}
\author{Nome do autor}
\supervisor{Nome do orientador}

\newacronym{vant}{VANT}{Veículo Aéreo Não Tripulado}
\newacronym{imu}{IMU}{\emph{Inertial Measurement Unit}}
\newacronym{uart}{UART}{\emph{Universal Asynchronous Receiver Transmitter}}

% Símbolos:
\newglossaryentry{angvel}{
  name = $\Omega$ ,
  description = Velocidade angular,
}
\newglossaryentry{linvel}{
  name = ${V}$ ,
  description = Velocidade linear,
}
\newglossaryentry{euler}{
  name = $\Phi$ ,
  description = Ângulos de Euler: \emph{roll} ($\phi$); \emph{pitch} ($\theta$) e \emph{yaw} ($\psi$),
}

\agradecimentos{Muito obrigado por tudo ter funcionado.}
\resumo{A página de resumo -- assim como de agradecimentos -- só aparece quando o comando \texttt{resumo} for posto no preâmbulo do arquivo, caso contrário essa página não é incluída}

\begin{document}

\maketitle % TODO: tem como dispensar esse comando, tipo redefinindo o ambiente "document"

%%%%%%%%%%%%%%%%%%%%%%%%%%%%
%% Início do documento
%%%%%%%%%%%%%%%%%%%%%%%%%%%%
\section{Introdução}

Este é o texto de introdução. Este arquivo serve para mostrar um pouco quais são as particularidades dos documentos com a classe \automatex.

\subsection{Usando figuras}

Os ambientes \verb+figure+ e \verb+table+ já são configurados para que estes insiram figuras e tabelas no ponto em que estas aparecem no texto. Por exemplo, se eu quero citar a Figura~\ref{fig:ee} aqui, ela deve ser inserida logo depois deste paragrafo com os comandos

\begin{minipage}{\textwidth}
  \begin{verbatim}
    \begin{figure}
      \includegraphics[width=.3\textwidth]{imagens/logo_eng}
      \caption{A Escola de Engenharia da UFRGS}
      \label{fig:ee}
      \source{\url{http://www.ufrgs.br/engenharia/}}
    \end{figure}
  \end{verbatim}
\end{minipage}


\begin{figure}
  \includegraphics[width=.3\textwidth]{imagens/logo_eng}
  \caption{A Escola de Engenharia da UFRGS}
  \label{fig:ee}
  \source{\url{http://www.ufrgs.br/engenharia/}}
\end{figure}

Além disto, o comando personalizado \verb+\source+ permite a citação da fonte da figura. O padrão é incluí-lo depois do comando \verb+\label+ para garantir que este não cause nenhum problema no sumário.

Da mesma forma que para as figuras, podemos citar tabelas, como a Tabela~\ref{tab:exemplo}. Como a legenda das tabelas deve aparecer sobre esta, o comando \verb+\caption+ deve aparecer antes do ambiente \texttt{tabular}.

\begin{table}
  \caption{Exemplo de tabela}
  \begin{tabular}{||c c c c||}
     \hline
     Col1 & Col2 & Col2 & Col3 \\ \hline\hline
     1 & 6 & 87837 & 787 \\ \hline
     2 & 7 & 78 & 5415 \\ \hline
     3 & 545 & 778 & 7507 \\ \hline
     4 & 545 & 18744 & 7560 \\ \hline
     5 & 88 & 788 & 6344 \\ \hline
  \end{tabular}
  \label{tab:exemplo}
\end{table}

\begin{minipage}{\textwidth}
  \begin{verbatim}
    \begin{table}
      \caption{Exemplo de tabela}
      \begin{tabular}{||c c c c||}
         \hline
         Col1 & Col2 & Col2 & Col3 \\ \hline\hline
         1 & 6 & 87837 & 787 \\ \hline
         2 & 7 & 78 & 5415 \\ \hline
         3 & 545 & 778 & 7507 \\ \hline
         4 & 545 & 18744 & 7560 \\ \hline
         5 & 88 & 788 & 6344 \\ \hline
      \end{tabular}
      \label{tab:exemplo}
    \end{table}
  \end{verbatim}
\end{minipage}

\subsection{Opção de língua}

A classe \verb+automatex+ aceita dois idiomas: português e inglês. Assim, é necessário definir no início do documento qual dos idiomas vai ser utilizado por
\begin{verbatim}
  \documentclass[portugues]{automatex}
\end{verbatim}
ou
\begin{verbatim}
  \documentclass[ingles]{automatex}
\end{verbatim}

\subsubsection{Equações}

Na de mais, apenas para mostrar como que ficam as referências para as equações. A equação de Navier-Stokes, em sua forma geral que eu copiei da Wikipédia, é dada por
\begin{equation}
  \rho\frac{D\mathbf{v}}{D t} = -\nabla p + \nabla \cdot\mathbb{T} + \rho\mathbf{f},
  \label{eq:navier-stokes}
\end{equation}
onde $\rho$, $\mathbf{v}$, $\mathbb{T}$ e $\mathbf{f}$ são variáveis que não vêm ao caso.

\subsection{Abreviaturas e símbolos}

Podemos usar todas as abreviaturas definidas no preâmbulo do documento como \gls{uart}, \gls{imu} e \gls{vant}. Da mesma forma podemos citar os símbolos \gls{angvel}, \gls{linvel} e \gls{euler}.

Para que o pacote de glossário funcione, é necessário compilar o arquivo \verb+.tex+, rodar o comando \verb+makeglossaries+, para enfim recompilar o arquivo \verb+.tex+.

\subsection{Citações bibliográficas}

Para respeitar o modelo das citações, é necessário usar o pacote \verb+natbib+ que disponibiliza os seguintes tipos de citação
\begin{itemize}
  \item \textbackslash{cite} \cite{small}.
  \item \textbackslash{citet} \citet{small}.
  \item \textbackslash{citep} \citep{small}.
  \item \textbackslash{citealt} \citealt{small}.
  \item \textbackslash{citealp} \citealp{small}.
\end{itemize}

Assim, se quisermos dizer que \citealp{big}, fez alguma coisa, temos que usar o comando \verb+\citealp+,
enquanto que para uma citação indireta usamos o comando \verb+\citep+ \citep{big}.

\subsection{Anexos e apêndices}

Os comandos \verb+\anexos+ e \verb+\apendices+ marcam o início das seções de anexos e apêndices, respectivamente. A partir destes comandos, deve-se usar apenas subseções e níveis inferiores. Anexos e apêndices são ordenados por letras A, B, C... Ou seja, a numeração é reiniciada toda vez que um desses comandos é inserido.

\subsection{Considerações finais}

Para configurar as informações da capa, temos quatro entradas básicas
\begin{enumerate}
  \item \verb+\title+: o título do trabalho
  \item \verb+\shorttitle+: o título resumido que aparece no cabeçalho das páginas
  \item \verb+\author+: autor do trabalho
  \item \verb+\supervisor+: o professor orientador
\end{enumerate}

\textbf{Importante:} não esquecer de rodar os comandos \verb+makeglossaries+ e \verb+bibtex+ para gerar o glossário e as referências bibliográficas (arquivo \verb+.bib+).

Por fim, para garantir que tudo seja devidamente referenciado (tanto bibligrafia, quanto listas) os comandos a serem executados devem ser algo do tipo\footnote{Exatamente, para que as listas de figuras, tabelas, etc., sejam geradas, é necessário compilar o arquivo duas vezes no final}
\begin{verbatim}
  pdflatex main
  makeglossaries main
  bibtex main.aux
  pdflatex main
  pdflatex main
\end{verbatim}

\section{Revisão bibliográfica}

\lipsum[10]

\subsection{Revisão do assunto 1}

\lipsum[11]

\subsubsection{Revisão do tópico 1}

\lipsum[12]

%% Bibliografia
\bibliography{biblio}

\end{document}
