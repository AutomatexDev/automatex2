% !TEX root = exemplo.tex
% !TEX program = lualatex

\documentclass{automatex}

\usepackage{lipsum}

\title{Título do trabalho de conclusão de curso}
\shorttitle{Título do trabalho de conclusão de curso}
\author{Nome do autor}
\supervisor{Nome do orientador}

% Abreviações
\newacronym{vant}{VANT}{Veículo Aéreo Não Tripulado}
\newacronym{imu}{IMU}{\emph{Inertial Measurement Unit}}
\newacronym{uart}{UART}{\emph{Universal Asynchronous Receiver Transmitter}}

% Símbolos:
\newglossaryentry{angvel}{
  name = $\Omega$ ,
  description = Velocidade angular,
}
\newglossaryentry{linvel}{
  name = ${V}$ ,
  description = Velocidade linear,
}
\newglossaryentry{euler}{
  name = $\Phi$ ,
  description = Ângulos de Euler: \emph{roll} ($\phi$); \emph{pitch} ($\theta$) e \emph{yaw} ($\psi$),
}

\begin{document}

\maketitle

% TODO: a numeração tem que estar definida no próprio .cls. Como fazer isso?
\pagenumbering{roman}
\setcounter{page}{1}

%%%%%%%%%%%%%%%%%%%%%%%%%%
%% Páginas iniciais
%%%%%%%%%%%%%%%%%%%%%%%%%%
\tableofcontents

\section*{Agradecimentos}
\addcontentsline{toc}{section}{\numberline{}Agradecimentos}
Obrigado.
\newpage

\addcontentsline{toc}{section}{\numberline{}Resumo}
\begin{abstract}
  Tudo resumido.
\end{abstract}
\newpage

\addcontentsline{toc}{section}{\numberline{}\listfigurename}
\listoffigures
\addcontentsline{toc}{section}{\numberline{}\listtablename}
\listoftables

\addcontentsline{toc}{section}{\numberline{}Lista de Abreviaturas e Siglas}
\printglossary[type=\acronymtype,title=Lista de Abreviaturas e Siglas]
\addcontentsline{toc}{section}{\numberline{}Lista de Simbolos}
\printglossary[title=Lista de Simbolos]

%%%%%%%%%%%%%%%%%%%%%%%%%%%%
%% Início do documentos
%%%%%%%%%%%%%%%%%%%%%%%%%%%%
\section{Introdução}

%% Define numeração de página 1,2,3
\pagenumbering{arabic}
\setcounter{page}{1}

\lipsum[1-2]
\begin{equation}
  a = \dfrac{1}{2}\;\int_x^y f(t)\; dt
\end{equation}

\lipsum[24-26]

\subsection{Motivações}

Aqui eu quero citar uma figura, por exemplo a Figura~\ref{fig:ufrgs}.

\begin{figure}
  \centering
  \includegraphics[width=.4\textwidth]{imagens/logo_ufrgs}
  \caption{UFRGS, a maior do Estado}
  \label{fig:ufrgs}
\end{figure}

\lipsum[13]

E assim eu cito a Figura~\ref{fig:ee}.

\begin{figure}
  \centering
  \includegraphics[width=.3\textwidth]{imagens/logo_eng}
  \caption{A Escola de Engenharia da UFRGS}
  \label{fig:ee}
\end{figure}


\section{Revisão bibliográfica}

\lipsum[10]

\subsection{Revisão do assunto 1}

\lipsum[11]

\subsubsection{Revisão do tópico 1}

\lipsum[12]

\section{Citações}

Para respeitar o modelo das citações, é necessário usar o pacote {\tt natbib} que disponibiliza
os seguintes tipos de citação
\begin{itemize}
  \item \textbackslash{cite} \cite{small}.
  \item \textbackslash{citet} \citet{small}.
  \item \textbackslash{citep} \citep{small}.
  \item \textbackslash{citealt} \citealt{small}.
  \item \textbackslash{citealp} \citealp{small}.
\end{itemize}

Assim, se quisermos dizer que \citealp{big}, fez alguma coisa, temos que usar o comando {\tt \textbackslash{citealp}},
enquanto que para uma citação indireta usamos o comando {\tt \textbackslash{citep}} \citep{big}.

%% Bibliografia
\bibliography{biblio}

\end{document}
